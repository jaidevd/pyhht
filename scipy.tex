% This is the source code for my presentation on pyhht at SciPy India 2011 held in Mumbai in Dec 2011

\documentclass[xcolor=dvipsnames]{beamer}
\usetheme{Warsaw}
\usecolortheme[named=Blue]{structure}
\usepackage{graphics}

\title{pyhht: A Python Toolbox for the Hilbert-Huang Transform}
\author{Jaidev Deshpande \\ deshpande.jaidev@gmail.com}
\institute{VIIT, Pune}
\date{December 5, 2011}


\begin{document}	

\begin{frame}
\titlepage
\end{frame}



\begin{frame}{Introduction - Why this talk?}
\begin{itemize}
\item Introducing HHT
\item An additional view of nonlinear and nonstationary phenomena
\item HHT is almost entirely algorithmic
\item It needs more math
\item I need an audience
\end{itemize}
\end{frame}

%3
\begin{frame}{Motivation for the HHT}
\begin{itemize}
\item What's the Hilbert Transform good for?
\begin{enumerate}
\item Analytical signals
\item Instantanous frequency
\end{enumerate}
\item Meaningful representations of data
\item A check against harmonics
\item Feature extraction
\end{itemize}
\end{frame}

%4
\begin{frame}{The Hilbert-Huang Transform}
\begin{itemize}
\item HHT = Empirical Mode Decomposition + Hilbert Spectral Analysis
\item EMD: breaks a wideband signal down into piecewise narrowband signals, called IMFs
\item Intrinsic Mode Functions:
\begin{enumerate}
\item Well behaved Hilbert Transforms
\item Piecewise stationary
\item Almost purely AM/FM
\end{enumerate}
\end{itemize}
\end{frame}

%5
\begin{frame}{Intrinsic Mode Functions}
\begin{definition}
A function is called an Intrinsic Mode Function if

\begin{enumerate}
\item the number of zero crossings and local extrema in the function differ at most by unity (takes care of localized oscillations)
\item the local mean of the enevelopes described by the local maxima and the local minima is zero at all times (required for meaningful instantaneous frequencies)
\end{enumerate}
\end{definition}

\end{frame}

%6
\begin{frame}{Empirical Mode Decomposition}
\begin{enumerate}
\item Find all local extrema in the signal
\item Join the maxima and minima with separate cubic splines, creating an upper and a lower envelope
\item Calculate the mean of the envelopes
\item Subtract mean from original signal
\item Repeat steps 1-4 until the result is an IMF
\item Subtract this IMF from the original signal
\item Repeat steps 1-6 till there are no more IMFs left in the signal
\end{enumerate}
\end{frame}
%
%%7
\begin{frame}{Hilbert Spectral Analysis}
Analytical signals from the IMFs:
\begin{equation}
z(t)=x(t)+i*y(t)
\end{equation}
Instantaneous phase can be calculated as:
\begin{equation}
\theta =  \tan ^{-1}  {\frac{y(t)}{x(t)}} 
\end{equation}
Instantaneous frequency, therefore:
\begin{equation}
\omega _{i}=\frac{d \theta}{dt}
\end{equation}
The original array is the real part of:
\begin{equation}
X(t)=\sum _{j=1} ^{N} x_{j}(t) \exp   (i\int \omega_{j} (t).dt)
\end{equation}
\end{frame}
%
%%8
\begin{frame}{Comparison with Fourier and Wavelet}
%\includegraphics[scale=0.3]{Screenshot}
\end{frame}
%
%%9
\begin{frame}{Heuristics for HHT}
\begin{itemize}
\item Stoppage criteria for sifting
\item Screening tools for IMFs based on statistical moments, information theoretic measures, etc
\item Alternatives to sifting like SVD, ICA
\item Better interpolation schemes
\end{itemize}
\end{frame}

\begin{frame}{What Would be Awesome}
\begin{itemize}
\item Speeding it up
\item A quantitative analysis of the EMD as a feature extraction, data compression method
\item Use of wavelets as a handle on the HHT
\item Establishing it as a veritable generalization of the Fourier method
\end{itemize}
\end{frame}
%
\begin{frame}
Thank You
\end{frame}

\end{document}